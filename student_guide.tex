\documentclass[aspectratio=169]{beamer}
\usepackage[utf8]{inputenc}
\usepackage{graphicx}
\usepackage{listings}
\usepackage{xcolor}
\usepackage{fontawesome}
\usepackage{tikz}
\usepackage{booktabs}
\usepackage{hyperref}

% Theme and colors
\usetheme{Madrid}
\usecolortheme{default}

% Custom colors
\definecolor{codegreen}{rgb}{0,0.6,0}
\definecolor{codegray}{rgb}{0.5,0.5,0.5}
\definecolor{codepurple}{rgb}{0.58,0,0.82}
\definecolor{backcolour}{rgb}{0.95,0.95,0.92}

% Code listing style
\lstdefinestyle{mystyle}{
    backgroundcolor=\color{backcolour},   
    commentstyle=\color{codegreen},
    keywordstyle=\color{blue},
    numberstyle=\tiny\color{codegray},
    stringstyle=\color{codepurple},
    basicstyle=\ttfamily\footnotesize,
    breakatwhitespace=false,         
    breaklines=true,                 
    captionpos=b,                    
    keepspaces=true,                 
    numbers=left,                    
    numbersep=5pt,                  
    showspaces=false,                
    showstringspaces=false,
    showtabs=false,                  
    tabsize=2
}

\lstset{style=mystyle}

% Title page information
\title[Statistical Webapp]{Building Your Statistical Webapp with AI}
\subtitle{Using AI Code Editors to Learn Statistics}
\author{Ka Wai Tsang}
\institute[CUHK(SZ)]{School of Data Science, CUHK(SZ)}
\date{September 17, 2025}

\begin{document}

% Title slide
\begin{frame}
\titlepage
\end{frame}

% Table of contents
\begin{frame}
\frametitle{Outline}
\tableofcontents
\end{frame}

\section{Introduction}

\begin{frame}
\frametitle{What We're Building}
\begin{columns}
\begin{column}{0.6\textwidth}
\begin{itemize}
\item \textbf{Multi-Modal Data Analysis Webapp}
\item Upload real datasets or generate synthetic data
\item AI-powered data creation with natural language
\item Basic statistics and MLE/MoM parameter estimation
\item Tab-based interface for different analysis methods
\item Ready for confidence intervals, ANOVA, regression
\end{itemize}
\end{column}
\begin{column}{0.4\textwidth}
\begin{center}
\textbf{Three Data Sources!} \\
\vspace{1em}
\textbf{Upload} \\
\textbf{Generate} \\
\textbf{AI Create}
\end{center}
\end{column}
\end{columns}
\end{frame}

\begin{frame}
\frametitle{Learning Objectives}
\begin{enumerate}
\item Master multi-modal data input (upload, generate, AI)
\item Understand MLE vs Method of Moments through interactive comparison
\item Learn tab-based UI design for statistical analysis
\item Implement parameter estimation algorithms
\item Connect theoretical concepts to practical applications
\item Prepare for advanced statistical methods (CI, ANOVA, regression)
\end{enumerate}
\end{frame}

\section{Why React + TypeScript for Data Analysis?}

\begin{frame}
\frametitle{The Challenge: Dynamic Data Analysis Webapp}
\begin{alertblock}{What We Need}
\begin{itemize}
\item \textbf{Modular Architecture}: Add new statistical methods easily
\item \textbf{Type Safety}: Prevent errors in statistical calculations
\item \textbf{Component Reusability}: Reuse UI elements across different analyses
\item \textbf{State Management}: Handle complex data flows and calculations
\item \textbf{Extensibility}: Add features without breaking existing code
\end{itemize}
\end{alertblock}

\begin{exampleblock}{Course Progression Example}
\begin{itemize}
\item Week 3: Distributions + MLE/MoM
\item Week 5: Confidence Intervals
\item Week 7: Hypothesis Testing
\item Week 9: ANOVA
\item Week 11: Linear Regression
\end{itemize}
\end{exampleblock}
\end{frame}

\begin{frame}
\frametitle{Why React is Perfect for Statistical Webapps}
\begin{columns}
\begin{column}{0.5\textwidth}
\textbf{Component-Based Architecture:}
\begin{itemize}
\item Each statistical method = separate component
\item Easy to add new analysis modules
\item Reusable UI elements (charts, inputs, results)
\item Independent development of features
\end{itemize}
\end{column}
\begin{column}{0.5\textwidth}
\textbf{State Management:}
\begin{itemize}
\item Centralized data handling
\item Real-time updates across components
\item Complex calculation workflows
\item User interaction management
\end{itemize}
\end{column}
\end{columns}

\begin{alertblock}{Real Example}
\begin{itemize}
\item \texttt{DistributionGenerator} component
\item \texttt{ConfidenceInterval} component (added later)
\item \texttt{ANOVAAnalysis} component (added later)
\item All share common \texttt{DataVisualization} component
\end{itemize}
\end{alertblock}
\end{frame}

\begin{frame}
\frametitle{Why TypeScript is Essential for Statistics}
\begin{alertblock}{Type Safety in Statistical Computing}
\begin{itemize}
\item \textbf{Data Types}: Ensure correct data structures for analysis
\item \textbf{Function Signatures}: Prevent parameter errors in calculations
\item \textbf{Interface Definitions}: Standardize statistical method APIs
\item \textbf{Error Prevention}: Catch mistakes before they affect results
\end{itemize}
\end{alertblock}

\begin{exampleblock}{TypeScript Benefits for Statistics}
\begin{itemize}
\item \texttt{interface StatisticalTest \{ data: number[]; alpha: number; \}}
\item \texttt{type DistributionType = 'normal' | 'exponential' | 'binomial'}
\item \texttt{function calculateMLE(data: number[]): EstimationResult}
\item Automatic error checking for statistical functions
\end{itemize}
\end{exampleblock}
\end{frame}

\begin{frame}
\frametitle{Extensibility: Adding New Statistical Methods}
\begin{alertblock}{Modular Design Pattern}
\begin{enumerate}
\item Create new component for each statistical method
\item Define TypeScript interfaces for data structures
\item Implement calculation functions with type safety
\item Add to main application routing
\item Reuse existing visualization components
\end{enumerate}
\end{alertblock}

\begin{exampleblock}{Adding Confidence Intervals Later}
\begin{itemize}
\item Create \texttt{ConfidenceIntervalComponent}
\item Define \texttt{ConfidenceIntervalData} interface
\item Implement \texttt{calculateCI()} function
\item Add to main app navigation
\item Reuse existing \texttt{ChartComponent}
\end{itemize}
\end{exampleblock}

\begin{alertblock}{AI Prompt for Extensions}
\texttt{"I want to add confidence interval analysis to my existing statistical webapp. How do I create a new component that integrates with my current React+TypeScript structure?"}
\end{alertblock}
\end{frame}

\begin{frame}
\frametitle{Modern Development Benefits}
\begin{columns}
\begin{column}{0.5\textwidth}
\textbf{Development Experience:}
\begin{itemize}
\item Hot reloading for instant feedback
\item Excellent AI code editor support
\item Rich ecosystem of libraries
\item Professional development tools
\end{itemize}
\end{column}
\begin{column}{0.5\textwidth}
\textbf{Performance \& Scalability:}
\begin{itemize}
\item Efficient rendering for large datasets
\item Component optimization
\item Lazy loading of analysis modules
\item Responsive design for all devices
\end{itemize}
\end{column}
\end{columns}

\begin{alertblock}{Industry Standard}
\begin{itemize}
\item Used by major data analysis platforms
\item Excellent documentation and community
\item Future-proof technology stack
\item Transferable skills for career development
\end{itemize}
\end{alertblock}
\end{frame}

\section{Multi-Modal Data Analysis}

\begin{frame}
\frametitle{Three Ways to Get Data}
\begin{alertblock}{Data Input Methods}
\begin{enumerate}
\item \textbf{Upload Dataset}: Real-world CSV/Excel files
\item \textbf{Generate Data}: Synthetic data from probability distributions
\item \textbf{AI-Generated Data}: Natural language data creation
\end{enumerate}
\end{alertblock}

\begin{exampleblock}{Why Multiple Input Methods?}
\begin{itemize}
\item \textbf{Real Data}: Upload actual datasets for authentic analysis
\item \textbf{Controlled Data}: Generate data with known parameters for learning
\item \textbf{Custom Data}: Use AI to create specific scenarios for analysis
\end{itemize}
\end{exampleblock}
\end{frame}

\begin{frame}
\frametitle{Upload Dataset Workflow}
\begin{alertblock}{Step-by-Step Process}
\begin{enumerate}
\item Choose "Upload Dataset" from data input options
\item Drag \& drop or browse for CSV/Excel files
\item System automatically detects numeric columns
\item Select which columns to analyze
\item View results in tabbed interface
\end{enumerate}
\end{alertblock}

\begin{exampleblock}{File Requirements}
\begin{itemize}
\item First row: column headers
\item At least one numeric column
\item Supported: CSV, Excel (.xlsx, .xls)
\item Maximum size: 10MB
\end{itemize}
\end{exampleblock}
\end{frame}

\begin{frame}
\frametitle{Generate Data Workflow}
\begin{alertblock}{Distribution-Based Generation}
\begin{enumerate}
\item Choose "Generate Data" from data input options
\item Select probability distributions (Normal, Exponential, etc.)
\item Configure parameters (mean, std dev, lambda, etc.)
\item Set sample size (10 to 1000 samples)
\item Generate multiple datasets simultaneously
\end{enumerate}
\end{alertblock}

\begin{exampleblock}{Educational Benefits}
\begin{itemize}
\item \textbf{Known Parameters}: Compare estimates to true values
\item \textbf{Distribution Shapes}: See how parameters affect data
\item \textbf{Sample Size Effects}: Understand impact of sample size
\end{itemize}
\end{exampleblock}
\end{frame}

\begin{frame}
\frametitle{AI-Generated Data Workflow}
\begin{alertblock}{Natural Language Data Creation}
\begin{enumerate}
\item Choose "AI-Generated Data" from data input options
\item Describe the data you want in plain English
\item AI parses your description and generates appropriate data
\item Automatically analyzes the generated data
\end{enumerate}
\end{alertblock}

\begin{exampleblock}{Example Prompts}
\begin{itemize}
\item "Generate 100 random heights of college students (normal distribution, mean 170cm, std dev 10cm)"
\item "Create 50 data points representing waiting times at a coffee shop (exponential distribution, lambda=0.5)"
\item "Generate test scores for 200 students with a mean of 75 and standard deviation of 15"
\end{itemize}
\end{exampleblock}
\end{frame}

\begin{frame}
\frametitle{Setting Up AI Data Generation in China}
\begin{alertblock}{Getting API Keys in China}
\begin{enumerate}
\item \textbf{Alibaba Cloud DashScope} (Recommended)
\begin{itemize}
\item Website: \texttt{https://dashscope.aliyuncs.com/}
\item Register for Alibaba Cloud account
\item Enable DashScope service
\item Get API key from console
\end{itemize}
\item \textbf{Baidu ERNIE} (Alternative)
\begin{itemize}
\item Website: \texttt{https://cloud.baidu.com/product/wenxinworkshop}
\item Good Chinese language support
\item Competitive pricing
\end{itemize}
\item \textbf{Zhipu AI} (Alternative)
\begin{itemize}
\item Website: \texttt{https://www.zhipuai.cn/}
\item Reliable service
\item Good documentation
\end{itemize}
\end{enumerate}
\end{alertblock}
\end{frame}

\begin{frame}
\frametitle{Configuring Your API Key}
\begin{alertblock}{Environment Variable Setup}
\begin{enumerate}
\footnotesize
\item Create \texttt{.env} file in your project root
\item Add your API key: \texttt{VITE\_DASHSCOPE\_API\_KEY=your-key-here}
\item Restart your development server
\item Test with the API test component
\end{enumerate}
\end{alertblock}

\begin{exampleblock}{AI Prompt for Setup}
\footnotesize
\texttt{"I'm in China and want to set up AI data generation for my statistical webapp. Can you help me configure the DashScope API key and test that it's working?"}
\end{exampleblock}

\begin{alertblock}{Troubleshooting}
\begin{itemize}
\footnotesize
\item If API fails, the webapp will use simulated data
\item Check browser console for error messages
\item Verify API key is correctly set in \texttt{.env} file
\item Ensure development server is restarted after setting environment variable
\end{itemize}
\end{alertblock}
\end{frame}

\section{Analysis Methods: Basic Statistics vs MLE/MoM}

\begin{frame}
\frametitle{Tab-Based Analysis Interface}
\begin{alertblock}{Two Analysis Tabs}
\begin{enumerate}
\item \textbf{Basic Statistics}: Descriptive statistics and data preview
\item \textbf{MLE/MoM}: Parameter estimation and model fitting
\end{enumerate}
\end{alertblock}

\begin{exampleblock}{Why Separate Tabs?}
\begin{itemize}
\item \textbf{Clear Organization}: Different analysis methods in separate views
\item \textbf{Focused Learning}: Concentrate on one method at a time
\item \textbf{Easy Extension}: Add new tabs for future methods (CI, ANOVA, etc.)
\item \textbf{Better UX}: Less cluttered interface
\end{itemize}
\end{exampleblock}
\end{frame}

\begin{frame}
\frametitle{Basic Statistics Tab}
\begin{alertblock}{What You'll See}
\begin{itemize}
\item \textbf{Data Preview}: Sample values and total count
\item \textbf{Summary Statistics}: Mean, Median, Standard Deviation, Variance
\item \textbf{Range Statistics}: Min, Max, Range
\item \textbf{Data Source Info}: Which dataset and columns are being analyzed
\end{itemize}
\end{alertblock}

\begin{exampleblock}{Learning Objectives}
\begin{itemize}
\item Understand descriptive statistics
\item Learn to interpret data summaries
\item Compare different datasets
\item Prepare for inferential statistics
\end{itemize}
\end{exampleblock}
\end{frame}

\begin{frame}
\frametitle{MLE/MoM Tab}
\begin{alertblock}{Parameter Estimation Process}
\begin{enumerate}
\item Select distribution model to fit (Normal, Exponential, etc.)
\item View MLE (Maximum Likelihood Estimation) results
\item View MoM (Method of Moments) results
\item Compare MLE vs MoM estimates side-by-side
\end{enumerate}
\end{alertblock}

\begin{exampleblock}{Educational Value}
\begin{itemize}
\item \textbf{Method Comparison}: See when MLE and MoM differ
\item \textbf{Parameter Interpretation}: Understand what each parameter means
\item \textbf{Model Selection}: Try different distributions on same data
\item \textbf{Statistical Theory}: Connect formulas to real calculations
\end{itemize}
\end{exampleblock}
\end{frame}

\begin{frame}
\frametitle{MLE vs Method of Moments}
\begin{alertblock}{Maximum Likelihood Estimation (MLE)}
\begin{itemize}
\item \textbf{Principle}: Find parameters that maximize likelihood function
\item \textbf{Advantages}: Asymptotically efficient, consistent
\item \textbf{Formula}: $\hat{\theta}_{MLE} = \arg\max_{\theta} L(\theta)$
\item \textbf{Example}: Normal distribution: $\hat{\mu} = \bar{x}$, $\hat{\sigma}^2 = \frac{1}{n}\sum(x_i - \bar{x})^2$
\end{itemize}
\end{alertblock}

\begin{alertblock}{Method of Moments (MoM)}
\begin{itemize}
\item \textbf{Principle}: Match sample moments to population moments
\item \textbf{Advantages}: Simple, intuitive, often closed-form
\item \textbf{Formula}: $E[X^k] = \frac{1}{n}\sum x_i^k$ for $k$-th moment
\item \textbf{Example}: Normal distribution: $\hat{\mu} = \bar{x}$, $\hat{\sigma}^2 = \bar{x^2} - \bar{x}^2$
\end{itemize}
\end{alertblock}
\end{frame}

\section{Getting Started with AI}

\begin{frame}
\frametitle{What You Need}
\begin{alertblock}{Great News!}
You already have everything you need! Just use your AI code editor (Cursor, VS Code, etc.)
\end{alertblock}

\begin{columns}
\begin{column}{0.5\textwidth}
\textbf{You Already Have:}
\begin{itemize}
\item Computer with internet
\item AI code editor (Cursor/VS Code)
\item Basic statistics knowledge from lectures
\item Curiosity to learn!
\end{itemize}
\end{column}
\begin{column}{0.5\textwidth}
\textbf{We'll Help You:}
\begin{itemize}
\item Use AI to build the webapp
\item Understand statistical concepts
\item Create visualizations
\item Deploy your project
\end{itemize}
\end{column}
\end{columns}
\end{frame}

\begin{frame}
\frametitle{How AI Will Help You}
\begin{exampleblock}{AI as Your Learning Partner}
\begin{itemize}
\item \textbf{Setup}: AI guides you through project creation
\item \textbf{Coding}: AI writes code while explaining concepts
\item \textbf{Learning}: AI explains statistical principles
\item \textbf{Troubleshooting}: AI helps fix problems
\item \textbf{Understanding}: AI answers "why" questions
\end{itemize}
\end{exampleblock}

\begin{alertblock}{Your Role}
\begin{itemize}
\item Ask good questions
\item Experiment with the webapp
\item Connect coding to statistics
\item Learn by doing
\end{itemize}
\end{alertblock}
\end{frame}

\begin{frame}
\frametitle{Step 1: Ask AI to Set Up Your Environment}
\begin{alertblock}{AI Prompt to Use}
\texttt{"I want to build a statistical webapp for learning probability distributions. I need help setting up a React project with TypeScript. Can you guide me through installing Node.js and creating the project structure?"}
\end{alertblock}

\begin{exampleblock}{What AI Will Help You With}
\begin{itemize}
\item Installing Node.js (if needed)
\item Creating a React project
\item Setting up TypeScript
\item Installing necessary packages
\item Explaining each step
\end{itemize}
\end{exampleblock}
\end{frame}

\begin{frame}
\frametitle{Step 1: Learning Tips}
\begin{alertblock}{Pro Tips for AI Interaction}
\begin{itemize}
\item Ask AI to explain \textbf{why} each step is needed
\item Don't just ask "what" to do, ask "why"
\item Request examples and analogies
\item Ask for connections to statistics concepts
\end{itemize}
\end{alertblock}

\begin{exampleblock}{Follow-Up Questions}
\begin{itemize}
\item \texttt{"Why do we use React for web development?"}
\item \texttt{"What is TypeScript and why is it useful?"}
\item \texttt{"How does this relate to what I learned in statistics class?"}
\end{itemize}
\end{exampleblock}
\end{frame}

\begin{frame}
\frametitle{Step 2: Ask AI About Project Structure}
\begin{alertblock}{AI Prompt to Use}
\texttt{"I'm creating a statistical webapp project. What should I name my project folder and where should I put it? I want to avoid common mistakes that beginners make."}
\end{alertblock}

\begin{exampleblock}{AI Will Explain}
\begin{itemize}
\item Good folder naming conventions
\item Where to place your project
\item How to organize files
\item Common mistakes to avoid
\end{itemize}
\end{exampleblock}
\end{frame}

\begin{frame}
\frametitle{Step 2: Understanding Project Organization}
\begin{alertblock}{Follow-Up Question}
\texttt{"Can you show me the basic folder structure I'll need for a React statistical webapp?"}
\end{alertblock}

\begin{exampleblock}{Learning Questions}
\begin{itemize}
\item \texttt{"Why is project organization important?"}
\item \texttt{"What does each folder contain?"}
\item \texttt{"How does this structure help with statistics visualization?"}
\end{itemize}
\end{exampleblock}
\end{frame}

\section{Building with AI}

\begin{frame}
\frametitle{Step 3: Let AI Create Your Project}
\begin{alertblock}{AI Prompt to Use}
\texttt{"I want to create a React webapp for statistics learning. Can you help me create the project using Vite with TypeScript? I'm a beginner, so please explain each command and what it does."}
\end{alertblock}

\begin{exampleblock}{AI Will Guide You Through}
\begin{itemize}
\item Creating the React project
\item Understanding what Vite is
\item Why we use TypeScript
\item Installing dependencies
\item Handling interactive prompts
\end{itemize}
\end{exampleblock}
\end{frame}

\begin{frame}
\frametitle{Step 3: When You Get Stuck}
\begin{alertblock}{Common AI Prompts}
\begin{itemize}
\item \texttt{"I'm getting an interactive prompt asking me to choose options. What should I select for a statistical webapp project?"}
\item \texttt{"The terminal is asking me questions I don't understand. Can you help?"}
\item \texttt{"What does this error message mean?"}
\end{itemize}
\end{alertblock}

\begin{exampleblock}{Learning Questions}
\begin{itemize}
\item \texttt{"What is Vite and why do we use it instead of other tools?"}
\item \texttt{"How does TypeScript help with statistical programming?"}
\end{itemize}
\end{exampleblock}
\end{frame}

\begin{frame}
\frametitle{Step 4: Ask AI About Styling and Charts}
\begin{alertblock}{AI Prompt to Use}
\texttt{"I want to make my statistical webapp look beautiful and add charts for data visualization. What packages should I install? I want to use Tailwind CSS for styling and need charts for histograms and statistical plots."}
\end{alertblock}

\begin{exampleblock}{AI Will Help You With}
\begin{itemize}
\item Installing Tailwind CSS (and why we use it)
\item Adding chart libraries (Recharts)
\item Installing icon libraries
\item Setting up configuration files
\item Troubleshooting installation issues
\end{itemize}
\end{exampleblock}
\end{frame}

\begin{frame}
\frametitle{Step 4: Understanding Visualization Tools}
\begin{alertblock}{Follow-Up Questions}
\begin{itemize}
\item \texttt{"I'm getting errors when installing packages. Can you help me troubleshoot and explain what each package does?"}
\item \texttt{"Why do we use Tailwind CSS instead of regular CSS?"}
\item \texttt{"What makes Recharts good for statistical visualization?"}
\end{itemize}
\end{alertblock}

\begin{exampleblock}{Learning Connections}
\begin{itemize}
\item \texttt{"How do these tools help me visualize statistical concepts?"}
\item \texttt{"What types of charts are best for different distributions?"}
\end{itemize}
\end{exampleblock}
\end{frame}

\section{Configuration with AI}

\begin{frame}
\frametitle{Step 5: Let AI Configure Your Styling}
\begin{alertblock}{AI Prompt to Use}
\texttt{"I need to configure Tailwind CSS for my statistical webapp. Can you help me set up the configuration file? I want a professional color scheme and need it to work with my React TypeScript project."}
\end{alertblock}

\begin{exampleblock}{AI Will Provide}
\begin{itemize}
\item Complete Tailwind configuration
\item Professional color palette
\item Explanation of each setting
\item How to customize colors later
\end{itemize}
\end{exampleblock}
\end{frame}

\begin{frame}
\frametitle{Step 5: Understanding Configuration}
\begin{alertblock}{Understanding Questions}
\begin{itemize}
\item \texttt{"Can you explain what each part of this configuration does and why it's important for my webapp?"}
\item \texttt{"How do colors affect user experience in statistical visualization?"}
\item \texttt{"What makes a color scheme professional for data analysis?"}
\end{itemize}
\end{alertblock}
\end{frame}

\begin{frame}
\frametitle{Step 6: Ask AI About CSS Setup}
\begin{alertblock}{AI Prompt to Use}
\texttt{"I need to set up the CSS file for my statistical webapp. Can you help me configure Tailwind CSS imports and create some custom styles? I want a clean, professional look with good typography."}
\end{alertblock}

\begin{exampleblock}{AI Will Help You Create}
\begin{itemize}
\item Tailwind CSS imports
\item Custom font setup
\item Reusable component styles
\item Professional color scheme
\item Responsive design classes
\end{itemize}
\end{exampleblock}
\end{frame}

\begin{frame}
\frametitle{Step 6: Understanding CSS and Design}
\begin{alertblock}{Learning Questions}
\begin{itemize}
\item \texttt{"Can you explain how Tailwind CSS works and why we use these specific classes for a professional webapp?"}
\item \texttt{"How does typography affect readability in data visualization?"}
\item \texttt{"What makes a design 'professional' for statistical applications?"}
\end{itemize}
\end{alertblock}
\end{frame}

\section{Testing Your Setup}

\begin{frame}
\frametitle{Step 7: Ask AI to Help You Test}
\begin{alertblock}{AI Prompt to Use}
\texttt{"I've set up my React project with Tailwind CSS. How do I start the development server and test that everything is working? What should I expect to see?"}
\end{alertblock}

\begin{exampleblock}{AI Will Guide You Through}
\begin{itemize}
\item Starting the development server
\item Understanding the output messages
\item Opening your webapp in the browser
\item What the default page should look like
\item How to verify Tailwind is working
\end{itemize}
\end{exampleblock}
\end{frame}

\begin{frame}
\frametitle{Step 7: Troubleshooting Your Setup}
\begin{alertblock}{If Something Goes Wrong}
\begin{itemize}
\item \texttt{"I'm getting an error when trying to start my webapp. Can you help me troubleshoot this issue?"}
\item \texttt{"The styling doesn't look right. What could be wrong?"}
\item \texttt{"My browser shows a blank page. What should I check?"}
\end{itemize}
\end{alertblock}

\begin{exampleblock}{Success Indicators}
\begin{itemize}
\item Development server starts without errors
\item Browser shows the React default page
\item Tailwind styles are applied
\item No console errors in browser
\end{itemize}
\end{exampleblock}
\end{frame}

\begin{frame}
\frametitle{When Things Go Wrong - Ask AI!}
\begin{alertblock}{Common AI Prompts for Troubleshooting}
\begin{itemize}
\item \texttt{"My webapp won't start. Here's the error message: [paste error]. Can you help me fix this?"}
\item \texttt{"The styling isn't working. What could be wrong with my Tailwind CSS setup?"}
\item \texttt{"I'm getting import errors. Can you help me understand what's happening?"}
\item \texttt{"My terminal is asking me questions I don't understand. What should I choose?"}
\end{itemize}
\end{alertblock}
\end{frame}

\begin{frame}
\frametitle{Getting Effective AI Help}
\begin{exampleblock}{Pro Tips for AI Help}
\begin{itemize}
\item Always copy and paste exact error messages
\item Tell AI what you were trying to do
\item Ask for explanations, not just fixes
\item Don't be afraid to ask "why" questions
\end{itemize}
\end{exampleblock}

\begin{alertblock}{Best Practices}
\begin{itemize}
\item Include context about your project
\item Mention what you've already tried
\item Ask for step-by-step guidance
\item Request explanations of concepts you don't understand
\end{itemize}
\end{alertblock}
\end{frame}

\section{Building the Statistical Features}

\begin{frame}
\frametitle{Step 8: Ask AI to Build Your Multi-Modal Data Analysis Webapp}
\begin{alertblock}{AI Prompt to Use}
\footnotesize
\texttt{"I want to build a comprehensive data analysis webapp with three data input methods: file upload, distribution generation, and AI-generated data. The webapp should have a tab-based interface with Basic Statistics and MLE/MoM analysis tabs. Can you help me create a React component structure that supports all these features?"}
\end{alertblock}

\begin{alertblock}{For Students in China}
\footnotesize
\texttt{"I'm in China and want to integrate real LLM APIs for AI data generation. Can you help me set up DashScope API integration with proper error handling and fallback to simulated data?"}
\end{alertblock}

\begin{exampleblock}{AI Will Help You Create}
\begin{itemize}
\footnotesize
\item Multi-modal data input system (upload, generate, AI); Tab-based analysis interface (Basic Stats, MLE/MoM)
\item File upload with column selection; Distribution generator with multiple datasets
\item AI data generator with natural language prompts;  MLE and Method of Moments parameter estimation
\end{itemize}
\end{exampleblock}
\end{frame}

\begin{frame}
\frametitle{Step 8: Learning Data Analysis Through Building}
\begin{alertblock}{Learning Approach}
Ask AI to explain the statistical concepts as you build:
\begin{itemize}
\footnotesize
\item \texttt{"Can you explain how MLE and Method of Moments work and when to use each?"}
\item \texttt{"How do I implement file upload with column detection in React?"}
\item \texttt{"What's the best way to structure a tab-based analysis interface?"}
\item \texttt{"How do I create an AI data generator that parses natural language?"}
\end{itemize}
\end{alertblock}

\begin{exampleblock}{Architecture Questions}
\begin{itemize}
\footnotesize
\item \texttt{"How can I design my webapp to handle multiple data sources seamlessly?"}
\item \texttt{"What TypeScript interfaces should I create for different data input methods?"}
\item \texttt{"How do I structure components to be easily extensible for new analysis methods?"}
\end{itemize}
\end{exampleblock}
\end{frame}

\begin{frame}
\frametitle{Step 9: Test and Learn with AI}
\begin{alertblock}{Testing Your Webapp}
\begin{enumerate}
\item Try different distributions
\item Adjust parameters and see what happens
\item Generate data and observe patterns
\item Ask AI: \texttt{"Why does changing the mean shift the distribution?"}
\item Ask AI: \texttt{"What happens when I increase the standard deviation?"}
\end{enumerate}
\end{alertblock}
\end{frame}

\begin{frame}
\frametitle{Step 9: Learning Questions to Ask AI}
\begin{exampleblock}{Data Analysis Understanding}
\begin{itemize}
\item \texttt{"Can you explain what these MLE and MoM estimation results mean?"}
\item \texttt{"How do I interpret the differences between MLE and MoM estimates?"}
\item \texttt{"What makes a good multi-modal data input interface?"}
\item \texttt{"How should I prepare my code for adding confidence intervals and hypothesis testing?"}
\end{itemize}
\end{exampleblock}

\begin{alertblock}{Connecting to Course Material}
\begin{itemize}
\item \texttt{"How does this multi-modal approach relate to real-world data analysis?"}
\item \texttt{"What statistical methods should I plan to add next (confidence intervals, ANOVA, regression)?"}
\item \texttt{"How can I extend this webapp for hypothesis testing with two datasets?"}
\end{itemize}
\end{alertblock}
\end{frame}

\section{Understanding AI API and Deployment}

\begin{frame}
\frametitle{AI Data Generation: How It Works}
\begin{alertblock}{Development Environment (Local)}
\begin{itemize}
\item Real AI API calls to DashScope
\item Actual AI-generated data
\item Full functionality for learning
\end{itemize}
\end{alertblock}

\begin{exampleblock}{Production Environment (Deployed)}
\begin{itemize}
\item Intelligent simulated data generation
\item Context-aware responses to prompts
\item Maintains educational value
\end{itemize}
\end{exampleblock}

\begin{alertblock}{For Students}
\begin{itemize}
\item Run locally to see real AI integration
\item Deployed version works perfectly for demos
\item Both environments provide learning value
\end{itemize}
\end{alertblock}
\end{frame}

\begin{frame}
\frametitle{Understanding AI API Limitations}
\begin{alertblock}{Important: Why AI API Has Limitations}
\begin{itemize}
\item \textbf{CORS Policy}: Browsers block cross-origin requests
\item \textbf{GitHub Pages}: Static hosting only, no server-side code
\item \textbf{Security Feature}: This is normal web behavior
\end{itemize}
\end{alertblock}

\begin{exampleblock}{Our Solution}
\begin{itemize}
\item Automatic detection of environment
\item Graceful fallback to simulated data
\item Clear communication to users
\item No broken functionality
\end{itemize}
\end{exampleblock}
\end{frame}

\begin{frame}
\frametitle{Advanced: Future Deployment Options}
\begin{alertblock}{When You're Ready for Real AI in Production}
\begin{itemize}
\item \textbf{Vercel}: Serverless functions can proxy API calls
\item \textbf{Netlify}: Similar serverless function support
\item \textbf{Custom Backend}: Your own server to handle API calls
\end{itemize}
\end{alertblock}

\begin{exampleblock}{For Now: GitHub Pages is Perfect}
\begin{itemize}
\item Simple deployment for beginners
\item No server configuration needed
\item Free hosting with automatic updates
\item Simulated data provides full functionality
\end{itemize}
\end{exampleblock}
\end{frame}

\section{Sharing Your Work}

\begin{frame}
\frametitle{Step 10: Ask AI to Help You Deploy}
\begin{alertblock}{AI Prompt to Use}
\texttt{"I've built my statistical webapp and want to share it online. Can you help me deploy it to GitHub Pages? I'm a beginner with Git and GitHub, so please explain each step."}
\end{alertblock}

\begin{exampleblock}{AI Will Guide You Through}
\begin{itemize}
\item Setting up a GitHub repository
\item Understanding Git basics
\item Uploading your code
\item Configuring GitHub Pages
\item Making your webapp live online
\item Understanding AI API limitations in production
\end{itemize}
\end{exampleblock}
\end{frame}

\begin{frame}
\frametitle{Step 10: Understanding Deployment}
\begin{alertblock}{Learning Questions}
\begin{itemize}
\item \texttt{"What is Git and why do we use it for web development?"}
\item \texttt{"How does GitHub Pages work and why is it useful?"}
\item \texttt{"What happens when I update my code?"}
\end{itemize}
\end{alertblock}

\begin{exampleblock}{Benefits of Online Deployment}
\begin{itemize}
\item Share your work with others
\item Access your webapp from anywhere
\item Demonstrate your learning
\item Build a portfolio
\end{itemize}
\end{exampleblock}
\end{frame}

\begin{frame}
\frametitle{Testing Your Webapp: Local vs Deployed}
\begin{alertblock}{Local Testing (Development)}
\begin{itemize}
\item Run: \texttt{npm run dev}
\item Test AI data generation with real API
\item Verify all features work correctly
\item Debug any issues before deployment
\end{itemize}
\end{alertblock}

\begin{exampleblock}{Deployed Testing (Production)}
\begin{itemize}
\item Visit your GitHub Pages URL
\item Test AI data generation (uses simulated data)
\item Verify all features work correctly
\item Check that user experience is smooth
\end{itemize}
\end{exampleblock}

\begin{alertblock}{Key Point}
Both environments should work perfectly - just with different AI data sources!
\end{alertblock}
\end{frame}

\begin{frame}
\frametitle{Step 11: Ask AI About GitHub Pages Setup}
\begin{alertblock}{AI Prompt to Use}
\texttt{"I've uploaded my code to GitHub. Now I want to make my statistical webapp live on GitHub Pages. Can you walk me through the settings and explain how the deployment process works?"}
\end{alertblock}

\begin{exampleblock}{AI Will Help You With}
\begin{itemize}
\item Navigating GitHub repository settings
\item Understanding GitHub Pages options
\item Setting up automatic deployment
\item Troubleshooting deployment issues
\item Understanding the deployment process
\end{itemize}
\end{exampleblock}
\end{frame}

\begin{frame}
\frametitle{Step 11: Understanding the Deployment Process}
\begin{alertblock}{Understanding Questions}
\begin{itemize}
\item \texttt{"Can you explain what happens when I push changes to GitHub and how my webapp gets updated?"}
\item \texttt{"Why do we need automatic deployment?"}
\item \texttt{"How long does it take for changes to appear online?"}
\end{itemize}
\end{alertblock}

\begin{exampleblock}{Success Indicators}
\begin{itemize}
\item Your webapp is accessible via a public URL
\item Changes appear online after pushing to GitHub
\item No deployment errors in GitHub Actions
\end{itemize}
\end{exampleblock}
\end{frame}

\section{AI-Powered Troubleshooting}

\begin{frame}
\frametitle{When You Get Stuck - Ask AI!}
\begin{alertblock}{Effective AI Troubleshooting Prompts}
\begin{itemize}
\footnotesize
\item \texttt{"I'm getting this error: [paste exact error]. What does it mean and how can I fix it?"}
\item \texttt{"My webapp was working yesterday but now it's broken. What could have changed?"}
\item \texttt{"I don't understand this error message. Can you explain it in simple terms?"}
\item \texttt{"I tried to follow the tutorial but something went wrong. Can you help me debug this?"}
\end{itemize}
\end{alertblock}

\begin{exampleblock}{Best Practices for AI Help}
\begin{itemize}
\footnotesize
\item Always include the exact error message
\item Describe what you were trying to do
\item Mention what you've already tried
\item Ask for explanations, not just solutions
\end{itemize}
\end{exampleblock}
\end{frame}

\begin{frame}
\frametitle{Learning with AI}
\begin{columns}
\begin{column}{0.5\textwidth}
\textbf{AI Learning Prompts:}
\begin{itemize}
\item \texttt{"Can you explain this statistical concept?"}
\item \texttt{"Why do we use this approach?"}
\item \texttt{"What's the difference between...?"}
\item \texttt{"How does this relate to what I learned in class?"}
\end{itemize}
\end{column}
\begin{column}{0.5\textwidth}
\textbf{AI Teaching Questions:}
\begin{itemize}
\item \texttt{"Can you give me examples of when to use this distribution?"}
\item \texttt{"What are the real-world applications?"}
\item \texttt{"How does this connect to other statistical concepts?"}
\end{itemize}
\end{column}
\end{columns}

\begin{alertblock}{Learning Strategy}
\begin{itemize}
\item Ask "why" questions, not just "how"
\item Connect coding concepts to statistics
\item Use AI to explain both technical and statistical concepts
\item Build understanding step by step
\end{itemize}
\end{alertblock}
\end{frame}

\section{What You've Accomplished}

\begin{frame}
\frametitle{Your Multi-Modal Data Analysis Webapp}
\begin{alertblock}{Skills Developed}
\begin{itemize}
\item Built a comprehensive multi-modal data analysis webapp
\item Mastered three data input methods (upload, generate, AI)
\item Implemented tab-based analysis interface
\item Learned MLE and Method of Moments parameter estimation
\item Created an extensible architecture for future statistical methods
\item Used AI tools effectively for both development and data generation
\end{itemize}
\end{alertblock}

\begin{exampleblock}{Ready for More}
Your webapp is now ready for:
\begin{itemize}
\item Confidence intervals (new tab)
\item Hypothesis testing (two-dataset comparison)
\item ANOVA analysis
\item Linear regression
\item Any future statistical methods!
\end{itemize}
\end{exampleblock}
\end{frame}

\begin{frame}
\frametitle{Key Features You've Built}
\begin{columns}
\begin{column}{0.5\textwidth}
\textbf{Data Input Methods:}
\begin{itemize}
\item File upload with column selection
\item Distribution-based data generation
\item AI-powered natural language data creation
\item Multi-dataset management
\end{itemize}
\end{column}
\begin{column}{0.5\textwidth}
\textbf{Analysis Capabilities:}
\begin{itemize}
\item Basic descriptive statistics
\item MLE parameter estimation
\item Method of Moments estimation
\item Side-by-side method comparison
\item Distribution model selection
\end{itemize}
\end{column}
\end{columns}

\begin{alertblock}{Educational Value}
\begin{itemize}
\item \textbf{Real Data}: Upload actual datasets for authentic analysis
\item \textbf{Controlled Learning}: Generate data with known parameters
\item \textbf{Method Comparison}: See MLE vs MoM differences
\item \textbf{Interactive Learning}: Change parameters and see immediate results
\end{itemize}
\end{alertblock}
\end{frame}

\section{Next Steps}

\begin{frame}
\frametitle{What's Next? Ask AI!}
\begin{alertblock}{AI Prompts for Extensions}
\begin{itemize}
\item \texttt{"How can I add confidence intervals as a third tab?"}
\item \texttt{"Can you help me add hypothesis testing with two datasets?"}
\item \texttt{"I want to add ANOVA analysis. Where do I start?"}
\item \texttt{"How can I add more distributions to my generator?"}
\item \texttt{"Can you help me integrate a real LLM API for data generation?"}
\end{itemize}
\end{alertblock}

\begin{exampleblock}{Learning-Driven Development}
\begin{itemize}
\item Ask AI to explain new statistical concepts as you learn them
\item Build new tabs for each statistical method covered in class
\item Connect webapp features to course material
\item Use AI to understand real-world applications
\item Extend the multi-modal approach to new data sources
\end{itemize}
\end{exampleblock}

\begin{alertblock}{Progressive Learning}
Start with what you have, then ask AI: \texttt{"Now that I have MLE/MoM working, what's the next logical statistical method to add to my webapp?"}
\end{alertblock}
\end{frame}

\begin{frame}
\frametitle{Ask AI to summarize}

\begin{alertblock}{AI-Assisted Reflection}
Ask AI: \texttt{"Can you help me write a summary of what I learned building this statistical webapp?"}
\end{alertblock}
\end{frame}


\begin{frame}
\begin{center}
\Huge \textbf{Start Building!}

\vspace{1em}

\Large Use AI to learn statistics through coding!

\vspace{2em}

\textbf{Remember:} 
\begin{itemize}
\item Ask AI lots of questions
\item Learn by doing
\item Connect coding to statistics
\item Don't be afraid to experiment!
\end{itemize}

\vspace{1em}

\textbf{Your AI is your best learning partner!}
\end{center}
\end{frame}

\end{document}
